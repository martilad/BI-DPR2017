\documentclass[11pt]{article}
\usepackage[a4paper,left=21mm,right=21mm,top=23mm,bottom=25mm]{geometry}
\usepackage{graphicx}
\usepackage[czech]{babel}
\usepackage[utf8]{inputenc}

\begin{document}
	\begin{titlepage}
	 \includegraphics[width=0.4\textwidth]{logo_FIT_cb}
		\begin{center}
        \vspace*{4.5cm}
        \huge
        \textbf{Evaluace algoritmů \\lokálně senzitivního hashování (LSH) \\v doporučovacích systémech}
        
        \vspace{3cm}
        \LARGE
        \textbf{ \textit{Ladislav Martínek}}

        \vfill
        \large
        
        Ročník: 3\\
        Obor: Znalostní inženýrství\\
        \vspace{0.5cm}
        
        Katedra aplikované matematiky\\
        Vedoucí práce: Ing. Tomáš Řehořek \\
        \vspace{2.5cm}
        Klíčová slova: aproximace algoritmu nejbližších sousedů, lokálně senzitivní hashování, doporučovací systémy, implementace testovací knihovny, experimenty a~analýza výsledků\\
        \vspace{0.5cm}
        \today
    \end{center}
 	\end{titlepage}
 
 

 
\section{Úvod} 
V~dnešní moderní společnosti, kde neustále roste objem dat nabízený na internetu,
je pro uživatele velmi těžké vyhledávat informace nebo produkty. Proto
se společnosti zabývají přizpůsobováním obsahu každému uživateli na internetu
např. doporučením, co by se uživateli mohlo líbit. Při doporučovaní se
snaží nabídnout např. film, článek, který by mohl uživatele zajímat. Doporučování
na internetu je nutné provádět v reálném čase, což se současnými
objemy dat je velice problémové. Z~tohoto důvodu je výhodné,
zároveň i~nutné se zabývat zrychlováním doporučovacích systému.

Existuje mnoho metod, které se zrychlováním zabývají. V~práci se věnuji
jedné konkrétní metodě a~to lokálně senzitivnímu hashování, pomocí kterého lze dosahovat několikanásobného zrychlení výpočtu při malé ztrátě přesnosti. Výstup mé práce by mel pomoci programátorům s~testováním a~vyhodnocováním různých parametrizací použité metody pro zrychlení vyhodnocování doporučovacích systémů.

Téma jsem si zvolil, neboť doporučovací systémy jsou využívány mnoha
společnostmi na internetových stránkách, kde nám jejich přítomnost usnadňuje
hledání informací. Nicméně je velmi výpočetně náročné vyhodnocovat
všechna nasbíraná data v~reálném čase, proto se v~práci zabývám aproximací doporučovacích systémů, konkrétně pak návrhem, implementací a~následným vyhodnocením knihovny na aproximaci hledání nejbližších sousedu.

\section{Cíl práce}
Cílem rešeršní části práce je získat přehled o doporučovacích systémech, seznámit
se s algoritmy nejbližších sousedu a nastudovat algoritmy lokálně senzitivního
hashování a možnosti využití těchto metod při aproximaci algoritmu
nejbližších sousedu. Dalším navazující cíl rešeršní části je představení metod
vyhodnocování úspěšnosti představených algoritmů a doporučovacích systému
založených na těchto algoritmech.

Cílem praktické části práce je navrhnout způsob aplikace metod lokálně
senzitivního hashování za účelem aproximace algoritmu hledání nejbližších
sousedu, poté návrh a implementace frameworku pro testování úspěšnosti
algoritmu lokálně senzitivního hashování v doporučovacích systémech. V neposlední řadě je cílem otestování implementace algoritmu s různou parametrizací na dvou datových sadách a diskuze výsledku, konkrétně poměr zrychlení výpočtu na úkor zhoršení přesnosti.

\section{Analýza}
Nejprve stručně vymezím pojmy a podle literatury popíši přístupy, kterým se v práci věnuji. Na to naváži analýzou současného stavu řešení aktuální problematiky.

\subsection{Základní pojmy}
Doporučovací systémy jsou nástroje a techniky, které generují doporučení položek, které by mohly být zajímavé nebo užitečné pro uživatele [1, pg.1]. Doporučením mohou být například produkty k prodeji, filmy nebo i slevové kupóny od obchodních řetězců.

Základním přístupem v doporučovacích systémech je filtrování obsahu a informací pomocí spolupráce mezi více uživateli (Kolaborativní filtování, angl. Collaborative filtering). V mé práci se budu věnovat filtrování pomocí spolupráce uživatelů. Klíčovou myšlenkou je, že pokud uživatelé sdílejí zájem o určité položky, budou tento zájem sdílet i nadále a budou preferovat podobné položky [4].

Algoritmy nejbližších sousedů (angl. k-nearest neighbors, k-NN) jsou přístupem kolaborativního filtrování a vždy fungují v konkrétním měřitelném prostoru, který muže být definován různými způsoby. Existence metriky je důležitá pro porovnávání podobnosti nebo vzdálenosti jednotlivých sousedů.

Lokálně senzitivní hashování (angl. Locality sensitive hashing, LSH) je metoda aproximace algoritmu k-NN. Podobně jako i jiné aproximační metody je LSH randomizované a náhodnost je typicky využita ke konstrukci datové struktury pro vytvoření modelu [13].
\subsection{Současný stav}

Co se řešilo\dots 
\section{Realizace}
Jak řeším\dots
\section{Závěr}
Cílem práce byl návrh a implementace knihovny pro testování metod lokálně senzitivního hashování, úspěšnosti těchto metod a jejich různých parametrizací.
	
	Knihovna implementována v rámci práce umožňuje testovat různé parametrizace a modely LSH na různých databázích, které lze specifikovat v konfiguračním souboru. Z výsledných dat generovaných frameworkem je možné pomocí nezávislého modulu generovat grafy pro lepší prezentaci výsledků.
	
	Dosažené výsledky při testování různých parametrizací jsou velice uspokojivé. Na dvou otestovaných datových sadách bylo zjištěno, že lze dosahovat času kolem 5 \%. Úspěšnost doporučování se při této aproximaci pohybuje okolo 97 - 99 \%. Při této úspěšnosti dosahují modely LSH přibližně o 20 \% lepší catalog coverage vzhledem k referenčnímu modelu. Doporučovací systém s metodami LSH tedy doporučuje větší množství různých položek, proto mohou být metody LSH užitečnější tím, že kromě časového zrychlení, mohou doporučovat větší část položek z databáze.
	
	Zajímavým poznatkem je zjištěná závislost mezi přesností na 10 nejbližších sousedů a úspěšnosti doporučování. Díky tomuto zjištění bylo možné otestovat velké množství různých parametrizací, protože testování přesnosti na 10 nejbližších sousedů je efektivnější. Testováním byli zjištěny některé hranice nastavení jednotlivých modelů a také nastavení pomocí kterého lze dosahovat optimálních výsledků.
	
	V neposlední řadě je zajímavý poznatek nastavení parametrů testu pro testování metod LSH pomocí frameworku. Pro každou databázi je nutné zvolit požadované parametry při testování z ohledem na velikost databáze. Tyto parametry lze podle počtu uživatelů lehce dopočítávat a tím přizpůsobit testování pro konkrétní databázi.
	
	V budoucnu by mohli tyto metody implementované jako součást doporučovacího systému, kde by bylo nutné se zabývat jejich paralelizací. Metody LSH lze velice efektivně paralelizovat obdobně jako samotný algoritmus k-NN. V budoucí práci by bylo také určitě nutné omezit maximální počet uživatelů v jednom bucketu.

Při využití v online doporučovacím systému by bylo nutné se zabývat automatizací nastavení modelu, tedy zvoleným množstvím hashovacích funkcí, množství bucketů a velikost okolí procházených bucketů. Zároveň na online datech není znám předem přesný objem dat ani počet uživatelů, proto by bylo zajímavé navrhnout způsob, kterým by bylo možné přizpůsobovat model při online běhu (např. přehashováním, přidávání nebo odebírání hashovacích funkcí). 
	
\section{Poznámky}

  \begin{enumerate}
\item Úvodní strana - název, jméno autora, jméno vedoucího práce, ročník a obor studia, případně logo fakulty
\item Klíčová slova - cca 5 až 10 odborných termínů charakterizujících zaměření bakalářské práce
\item Úvod 
\item Definice cíle(ů) bakalářské práce
\item Analýza současného stavu řešení problému - kdo a jakým způsobem již problém řešil, k čemu došel, výhody a nevýhody řešní
\item Aktuální stav řešení - řešení, které student zvolil
\item Závěr - míra splnění cíle(ů)
\item Seznam literárních zdrojů dle ČSN ISO 690
  \end{enumerate}
 
test\cite{guide}
\clearpage
\bibliographystyle{csn690}
\bibliography{zdroj}
\end{document}